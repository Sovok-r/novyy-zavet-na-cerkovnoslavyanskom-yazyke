
\section[Посла́нїе пе́рвое къ тїмоѳе́ю ст҃а́гѡ а҆пⷭ҇ла па́ѵла.]{ПОСЛА́НЇЕ ПЕ́РВОЕ КЪ ТЇМОѲЕ́Ю\\СТ҃А́ГѠ А҆Пⷭ҇ЛА ПА́ѴЛА.}

\subsection{Глава̀~\cuNum{1}. Зача́ло~\cuNum{278}.}

\cuLettrine Па́ѵелъ, посла́нникъ і҆и҃съ хрⷭ҇то́въ по повелѣ́нїю бг҃а сп҃са на́шегѡ и҆ гдⷭ҇а і҆и҃са хрⷭ҇та̀ ᲂу҆пова́нїѧ на́шегѡ, \textsuperscript{\cuNum{2}}\,тїмоѳе́ю при́сномꙋ ча́дꙋ въ вѣ́рѣ: блгⷣть, млⷭ҇ть, ми́ръ ѿ бг҃а ѻ҆ц҃а̀ на́шегѡ и҆ хрⷭ҇та̀ і҆и҃са гдⷭ҇а на́шегѡ. \textsuperscript{\cuNum{3}}\,Ꙗ҆́коже ᲂу҆моли́хъ тѧ̀ пребы́ти во є҆фе́сѣ, и҆ды́й въ македо́нїю, да завѣща́еши нѣ̑кимъ не и҆́накѡ ᲂу҆чи́ти, \textsuperscript{\cuNum{4}}\,нижѐ внима́ти ба́снемъ и҆ родосло́вїємъ безконє́чнымъ, ꙗ҆̀же стѧза̑нїѧ творѧ́тъ па́че, не́жели бж҃їе строе́нїе, є҆́же въ вѣ́рѣ. \textsuperscript{\cuNum{5}}\,Коне́цъ же завѣща́нїѧ є҆́сть любы̀ ѿ чи́ста се́рдца и҆ со́вѣсти бл҃гі́ѧ и҆ вѣ́ры нелицемѣ́рныѧ: \textsuperscript{\cuNum{6}}\,въ ни́хже нѣ́цыи погрѣши́вше, ᲂу҆клони́шасѧ въ сꙋеслѡ́вїѧ, \textsuperscript{\cuNum{7}}\,хотѧ́ще бы́ти законоꙋчи́тели, не разꙋмѣ́юще ни ꙗ҆̀же глаго́лютъ, ни ѡ҆ ни́хже ᲂу҆твержда́ютъ.

(Заⷱ҇\:\cuNum{279}.)~\textsuperscript{\cuNum{8}}\,Вѣ́мы же, ꙗ҆́кѡ до́бръ зако́нъ є҆́сть, а҆́ще кто̀ є҆го̀ зако́ннѣ твори́тъ, \textsuperscript{\cuNum{9}}\,вѣ́дый сїѐ, ꙗ҆́кѡ првⷣникꙋ зако́нъ не лежи́тъ\footnote{не положе́нъ}, но беззакѡ́ннымъ и҆ непокори̑вымъ, нечести̑вымъ же и҆ грѣ́шникѡмъ, непра́вєднымъ и҆ сквє́рнымъ, ѻ҆тца̀ и҆ ма́тере досади́телємъ, мꙋжеꙋбі́йцамъ, \textsuperscript{\cuNum{10}}\,блꙋдникѡ́мъ, мꙋжело́жникѡмъ, разбо́йникѡмъ, клеветникѡ́мъ, скотоло́жникѡмъ, лжи̑вымъ, клѧтвопрестꙋ́пникѡмъ, и҆ а҆́ще что̀ и҆́но здра́вомꙋ ᲂу҆че́нїю проти́витсѧ, \textsuperscript{\cuNum{11}}\,по бл҃говѣ́стїю сла́вы бл҃же́ннагѡ бг҃а, є҆́же мнѣ̀ ᲂу҆вѣ́рено бы́сть.

(Заⷱ҇\:\cuNum{280}.)~\textsuperscript{\cuNum{12}}\,И҆ благодарю̀ ᲂу҆крѣплѧ́ющаго мѧ̀ хрⷭ҇та̀ і҆и҃са гдⷭ҇а на́шего, ꙗ҆́кѡ вѣ́рна мѧ̀ непщева̀, положи́въ мѧ̀ въ слꙋ́жбꙋ, \textsuperscript{\cuNum{13}}\,бы́вша мѧ̀ и҆ногда̀ хꙋ́лника и҆ гони́телѧ и҆ досади́телѧ: но поми́лованъ бы́хъ, ꙗ҆́кѡ невѣ́дый сотвори́хъ въ невѣ́рствїи: \textsuperscript{\cuNum{14}}\,ᲂу҆преꙋмно́жисѧ же блгⷣть гдⷭ҇а на́шегѡ (і҆и҃са хрⷭ҇та̀) съ вѣ́рою и҆ любо́вїю ꙗ҆́же ѡ҆ хрⷭ҇тѣ̀ і҆и҃сѣ.

(Заⷱ҇.)~\textsuperscript{\cuNum{15}}\,Вѣ́рно сло́во и҆ всѧ́кагѡ прїѧ́тїѧ досто́йно, ꙗ҆́кѡ хрⷭ҇то́съ і҆и҃съ прїи́де въ мі́ръ грѣ́шники спⷭ҇тѝ, ѿ ни́хже пе́рвый є҆́смь а҆́зъ. \textsuperscript{\cuNum{16}}\,Но сегѡ̀ ра́ди поми́лованъ бы́хъ, да во мнѣ̀ пе́рвомъ пока́жетъ і҆и҃съ хрⷭ҇то́съ всѐ долготерпѣ́нїе, за ѻ҆́бразъ хотѧ́щихъ вѣ́ровати є҆мꙋ̀ въ жи́знь вѣ́чнꙋю. \textsuperscript{\cuNum{17}}\,Цр҃ю́ же вѣкѡ́въ нетлѣ́нномꙋ, неви́димомꙋ, є҆ди́номꙋ премꙋ́дромꙋ бг҃ꙋ, че́сть и҆ сла́ва во вѣ́ки вѣкѡ́въ. А҆ми́нь.

(Заⷱ҇\:\cuNum{281}.)~\textsuperscript{\cuNum{18}}\,Сїе́ же завѣща́нїе предаю́ ти, ча́до тїмоѳе́е, по бы́вшихъ на тѧ̀ пре́жде прⷪ҇ро́чествїихъ, да во́инствꙋеши въ ни́хъ до́брое во́инство, \textsuperscript{\cuNum{19}}\,и҆мѣ́ѧ вѣ́рꙋ и҆ бл҃гꙋ́ю со́вѣсть, ю҆́же нѣ́цыи ѿри́нꙋвше, ѿ вѣ́ры ѿпадо́ша: \textsuperscript{\cuNum{20}}\,ѿ ни́хже є҆́сть ѵ҆мене́й и҆ а҆леѯа́ндръ, и҆̀хже преда́хъ сатанѣ̀, да нака́жꙋтсѧ не хꙋ́лити.

\subsection{Глава̀~\cuNum{2}. Зача́ло~\cuNum{282}.}

\cuLettrine Молю̀ ᲂу҆̀бо пре́жде всѣ́хъ твори́ти моли̑твы, молє́нїѧ, прошє́нїѧ, благодарє́нїѧ за всѧ̑ человѣ́ки, \textsuperscript{\cuNum{2}}\,за царѧ̀ и҆ за всѣ́хъ, и҆̀же во вла́сти сꙋ́ть, да ти́хое и҆ безмо́лвное житїѐ поживе́мъ во всѧ́комъ бл҃гоче́стїи и҆ чистотѣ̀. \textsuperscript{\cuNum{3}}\,Сїе́ бо добро̀ и҆ прїѧ́тно пред̾ сп҃си́телемъ на́шимъ бг҃омъ, \textsuperscript{\cuNum{4}}\,и҆́же всѣ̑мъ человѣ́кѡмъ хо́щетъ спасти́сѧ и҆ въ ра́зꙋмъ и҆́стины прїитѝ. \textsuperscript{\cuNum{5}}\,Є҆ди́нъ бо є҆́сть бг҃ъ, и҆ є҆ди́нъ хода́тай бг҃а и҆ человѣ́кѡвъ, чл҃вѣ́къ хрⷭ҇то́съ і҆и҃съ, \textsuperscript{\cuNum{6}}\,да́вый себѐ и҆збавле́нїе за всѣ́хъ: свидѣ́телство времены̀ свои́ми, \textsuperscript{\cuNum{7}}\,въ не́же поста́вленъ бы́хъ а҆́зъ проповѣ́дникъ и҆ а҆пⷭ҇лъ, и҆́стинꙋ глаго́лю ѡ҆ хрⷭ҇тѣ̀, не лгꙋ̀, ᲂу҆чи́тель ꙗ҆зы́кѡвъ въ вѣ́рѣ и҆ и҆́стинѣ. \textsuperscript{\cuNum{8}}\,Хощꙋ̀ ᲂу҆̀бо, да моли̑твы творѧ́тъ мꙋ́жїе на всѧ́комъ мѣ́стѣ, воздѣ́юще прпⷣбныѧ рꙋ́ки без̾ гнѣ́ва и҆ размышле́нїѧ: \textsuperscript{\cuNum{9}}\,та́кожде и҆ жєны̀ во ᲂу҆краше́нїи лѣ́потномъ, со стыдѣ́нїемъ и҆ цѣломꙋ́дрїемъ да ᲂу҆краша́ютъ себѐ не въ плете́нїихъ, ни зла́томъ, и҆лѝ би́серми, и҆лѝ ри́зами многоцѣ́нными, \textsuperscript{\cuNum{10}}\,но, є҆́же подоба́етъ жена́мъ ѡ҆бѣщава́ющымсѧ бл҃гоче́стїю, дѣ́лы бл҃ги́ми. \textsuperscript{\cuNum{11}}\,Жена̀ въ безмо́лвїи да ᲂу҆чи́тсѧ со всѧ́кимъ покоре́нїемъ: \textsuperscript{\cuNum{12}}\,женѣ́ же ᲂу҆чи́ти не повелѣва́ю, нижѐ владѣ́ти мꙋ́жемъ, но бы́ти въ безмо́лвїи. \textsuperscript{\cuNum{13}}\,А҆да́мъ бо пре́жде со́зданъ бы́сть, пото́мъ же є҆́ѵа: \textsuperscript{\cuNum{14}}\,и҆ а҆да́мъ не прелсти́сѧ, жена́ же прелсти́вшисѧ, въ престꙋпле́нїи бы́сть: \textsuperscript{\cuNum{15}}\,спасе́тсѧ же чадоро́дїѧ ра́ди\footnote{чрез̾ чадоро́дїе}, а҆́ще пребꙋ́детъ въ вѣ́рѣ и҆ любвѝ и҆ во ст҃ы́ни съ цѣломꙋ́дрїемъ.

\subsection{Глава̀~\cuNum{3}. Зача́ло~\cuNum{283}.}

\cuLettrine Врно сло́во: а҆́ще кто̀ є҆пі́скопства хо́щетъ, добра̀ дѣ́ла жела́етъ. \textsuperscript{\cuNum{2}}\,Подоба́етъ ᲂу҆̀бо є҆пі́скопꙋ бы́ти непоро́чнꙋ, є҆ди́ныѧ жены̀ мꙋ́жꙋ, тре́звенꙋ, цѣломꙋ́дрꙋ, благоговѣ́йнꙋ, че́стнꙋ, страннолюби́вꙋ, ᲂу҆чи́телнꙋ: \textsuperscript{\cuNum{3}}\,не пїѧ́ницѣ, не бі́йцѣ, не сварли́вꙋ, не мшелои́мцꙋ, но кро́ткꙋ, (не зави́стливꙋ,) не сребролю́бцꙋ: \textsuperscript{\cuNum{4}}\,сво́й до́мъ до́брѣ пра́вѧщꙋ, ча̑да и҆мꙋ́щꙋ въ послꙋша́нїи со всѧ́кою чистото́ю\footnote{че́стностїю}: \textsuperscript{\cuNum{5}}\,а҆́ще же кто̀ своегѡ̀ до́мꙋ не ᲂу҆мѣ́етъ пра́вити, ка́кѡ ѡ҆ цр҃кви бж҃їей прилѣжа́ти возмо́жетъ; \textsuperscript{\cuNum{6}}\,Не новокреще́ннꙋ, да не разгордѣ́всѧ въ сꙋ́дъ впаде́тъ дїа́воль. \textsuperscript{\cuNum{7}}\,Подоба́етъ же є҆мꙋ̀ и҆ свидѣ́телство добро̀ и҆мѣ́ти ѿ внѣ́шнихъ, да не въ поноше́нїе впаде́тъ и҆ въ сѣ́ть непрїѧ́зненꙋ. \textsuperscript{\cuNum{8}}\,Дїа́конѡмъ та́кожде чи̑стымъ\footnote{честны́мъ}, не двоѧзы̑чнымъ, не вїнꙋ̀ мно́гꙋ внима́ющымъ, не скверностѧжа́тєлнымъ, \textsuperscript{\cuNum{9}}\,и҆мꙋ́щымъ та́инство вѣ́ры въ чи́стѣй со́вѣсти. \textsuperscript{\cuNum{10}}\,И҆ сі́и ᲂу҆́бѡ да и҆скꙋша́ютсѧ пре́жде, пото́мъ же да слꙋ́жатъ, непоро́чни сꙋ́ще. \textsuperscript{\cuNum{11}}\,Жена́мъ та́кожде чи̑стымъ\footnote{честны́мъ}, не клевети̑вымъ, (не нава́дницамъ,) тре́звєннымъ, вѣ̑рнымъ во все́мъ. \textsuperscript{\cuNum{12}}\,Дїа́кони да быва́ютъ є҆ди́ныѧ жены̀ мꙋ́жїе, ча̑да до́брѣ пра́вѧще и҆ своѧ̑ до́мы. \textsuperscript{\cuNum{13}}\,И҆́бо слꙋжи́вшїи до́брѣ степе́нь себѣ̀ до́бръ сниска́ютъ и҆ мно́гое дерзнове́нїе въ вѣ́рѣ, ꙗ҆́же ѡ҆ хрⷭ҇тѣ̀ і҆и҃сѣ.

(Заⷱ҇\:\cuNum{284}.)~\textsuperscript{\cuNum{14}}\,Сїѧ̑ пишꙋ̀ тебѣ̀, ᲂу҆пова́ѧ прїитѝ къ тебѣ̀ ско́рѡ: \textsuperscript{\cuNum{15}}\,а҆́ще же заме́длю, да ᲂу҆вѣ́си, ка́кѡ подоба́етъ въ домꙋ̀ бж҃їи жи́ти, ꙗ҆́же є҆́сть цр҃ковь бг҃а жи́ва, сто́лпъ и҆ ᲂу҆твержде́нїе и҆́стины. \textsuperscript{\cuNum{16}}\,И҆ и҆сповѣ́дꙋемѡ ве́лїѧ є҆́сть бл҃гоче́стїѧ та́йна: бг҃ъ ꙗ҆ви́сѧ во пло́ти, ѡ҆правда́сѧ въ дс҃ѣ, показа́сѧ а҆́гг҃лѡмъ, проповѣ́данъ бы́сть во ꙗ҆зы́цѣхъ, вѣ́ровасѧ въ мі́рѣ, вознесе́сѧ во сла́вѣ.

\subsection{Глава̀~\cuNum{4}.}

\cuLettrine Дх҃ъ же ꙗ҆́вственнѣ гл҃етъ, ꙗ҆́кѡ въ послѣ̑днѧѧ времена̀ ѿстꙋ́пѧтъ нѣ́цыи ѿ вѣ́ры, вне́млюще дꙋховѡ́мъ ле́стчымъ и҆ ᲂу҆че́нїємъ бѣсѡ́вскимъ, \textsuperscript{\cuNum{2}}\,въ лицемѣ́рїи лжесловє́сникъ, сожже́нныхъ свое́ю со́вѣстїю, \textsuperscript{\cuNum{3}}\,возбранѧ́ющихъ жени́тисѧ, ᲂу҆далѧ́тисѧ ѿ бра́шенъ, ꙗ҆̀же бг҃ъ сотворѝ въ снѣде́нїе со благодаре́нїемъ вѣ̑рнымъ и҆ позна́вшымъ и҆́стинꙋ.

(Заⷱ҇\:\cuNum{285}.)~\textsuperscript{\cuNum{4}}\,Занѐ всѧ́кое созда́нїе бж҃їе добро̀, и҆ ничто́же ѿме́тно, со благодаре́нїемъ прїе́млемо: \textsuperscript{\cuNum{5}}\,ѡ҆сщ҃а́етсѧ бо сло́вомъ бж҃їимъ и҆ моли́твою. \textsuperscript{\cuNum{6}}\,Сїѧ̑ всѧ̑ сказꙋ́ѧ бра́тїи, до́бръ бꙋ́деши слꙋжи́тель і҆и҃са хрⷭ҇та̀, пита́емь словесы̀ вѣ́ры и҆ до́брымъ ᲂу҆че́нїемъ, є҆мꙋ́же послѣ́довалъ є҆сѝ. \textsuperscript{\cuNum{7}}\,Скве́рныхъ же и҆ ба́бїихъ ба́сней ѿрица́йсѧ: ѡ҆бꙋча́й же себѐ ко бл҃гоче́стїю. \textsuperscript{\cuNum{8}}\,Тѣле́сное бо ѡ҆бꙋче́нїе вма́лѣ є҆́сть поле́зно, а҆ бл҃гоче́стїе на всѐ поле́зно є҆́сть, ѡ҆бѣтова́нїе и҆мѣ́ющее живота̀ ны́нѣшнѧгѡ и҆ грѧдꙋ́щагѡ.

(Заⷱ҇.)~\textsuperscript{\cuNum{9}}\,Вѣ́рно сло́во и҆ всѧ́кагѡ прїѧ́тїѧ досто́йно: \textsuperscript{\cuNum{10}}\,на сїе́ бо и҆ трꙋжда́емсѧ и҆ поноша́еми є҆смы̀, ꙗ҆́кѡ ᲂу҆пова́хомъ на бг҃а жи́ва, и҆́же є҆́сть сп҃си́тель всѣ̑мъ человѣ́кѡмъ, па́че же вѣ̑рнымъ. \textsuperscript{\cuNum{11}}\,Завѣщава́й сїѧ̑ и҆ ᲂу҆чѝ. \textsuperscript{\cuNum{12}}\,Никто́же ѡ҆ ю҆́ности твое́й да неради́тъ: но ѻ҆́бразъ бꙋ́ди вѣ̑рнымъ сло́вомъ, житїе́мъ, любо́вїю, дх҃омъ, вѣ́рою, чистото́ю. \textsuperscript{\cuNum{13}}\,До́ндеже прїидꙋ̀, внемлѝ чте́нїю, ᲂу҆тѣше́нїю, ᲂу҆че́нїю. \textsuperscript{\cuNum{14}}\,Не нерадѝ ѡ҆ свое́мъ дарова́нїи живꙋ́щемъ въ тебѣ̀, є҆́же дано̀ тебѣ̀ бы́сть прⷪ҇ро́чествомъ съ возложе́нїемъ рꙋ́къ свѧще́нничества. \textsuperscript{\cuNum{15}}\,Въ си́хъ поꙋча́йсѧ, въ си́хъ пребыва́й, (въ си́хъ разꙋмѣва́й:) да преспѣ́ѧнїе твоѐ ꙗ҆вле́но бꙋ́детъ во всѣ́хъ. \textsuperscript{\cuNum{16}}\,Внима́й себѣ̀ и҆ ᲂу҆че́нїю: и҆ пребыва́й въ ни́хъ. Сїѧ̑ бо творѧ̀, и҆ са́мъ спасе́шисѧ и҆ послꙋ́шающїи тебѐ.

\subsection{Глава̀~\cuNum{5}. Зача́ло~\cuNum{285}.}

\cuLettrine Ста́рцꙋ не творѝ па́кости, но ᲂу҆тѣша́й\footnote{ста́рца не ᲂу҆корѧ́й, но ᲂу҆молѧ́й} ꙗ҆́коже ѻ҆тца̀: ю҆́ношы, ꙗ҆́коже бра́тїю: \textsuperscript{\cuNum{2}}\,ста̑рицы, ꙗ҆́коже ма́тєри: ю҆́ныѧ, ꙗ҆́коже сєстры̀, со всѧ́кою чистото́ю. \textsuperscript{\cuNum{3}}\,Вдови̑цы чтѝ сꙋ́щыѧ и҆́стинныѧ вдови̑цы. \textsuperscript{\cuNum{4}}\,А҆́ще же ка́ѧ вдови́ца ча̑да и҆лѝ внꙋ́чата и҆́мать, да ᲂу҆ча́тсѧ пре́жде сво́й до́мъ бл҃гочести́вѡ ᲂу҆стро́ити и҆ взае́мъ воздаѧ́ти роди́телємъ: сїе́ бо є҆́сть бл҃гоꙋго́дно пред̾ бг҃омъ. \textsuperscript{\cuNum{5}}\,А҆ сꙋ́щаѧ и҆́стиннаѧ вдови́ца и҆ ᲂу҆едине́на, ᲂу҆пова́етъ на бг҃а и҆ пребыва́етъ въ моли́твахъ и҆ моле́нїихъ де́нь и҆ но́щь: \textsuperscript{\cuNum{6}}\,пита́ющаѧсѧ же простра́ннѡ, жива̀ ᲂу҆мерла̀. \textsuperscript{\cuNum{7}}\,И҆ сїѧ̑ завѣщава́й, да непоро́чни бꙋ́дꙋтъ. \textsuperscript{\cuNum{8}}\,А҆́ще же кто̀ ѡ҆ свои́хъ, па́че же ѡ҆ при́сныхъ\footnote{ѡ҆ дома́шнихъ} не промышлѧ́етъ, вѣ́ры ѿве́рглсѧ є҆́сть и҆ невѣ́рнагѡ го́ршїй є҆́сть. \textsuperscript{\cuNum{9}}\,Вдови́ца же да причита́етсѧ не ме́нши лѣ́тъ шести́десѧтихъ, бы́вши є҆ди́номꙋ мꙋ́жꙋ жена̀, \textsuperscript{\cuNum{10}}\,въ дѣ́лѣхъ до́брыхъ свидѣ́телствꙋема, а҆́ще ча̑да воспита́ла є҆́сть, а҆́ще ст҃ы́хъ но́зѣ ᲂу҆мы̀, а҆́ще стра̑нныѧ прїѧ́тъ, а҆́ще скѡ́рбнымъ ᲂу҆тѣше́нїе бы́сть\footnote{ско́рбныхъ снабдѣ̀}, а҆́ще всѧ́комꙋ дѣ́лꙋ бл҃гꙋ послѣ́довала є҆́сть.

(Заⷱ҇\:\cuNum{286}.)~\textsuperscript{\cuNum{11}}\,Ю҆́ныхъ же вдови́цъ ѿрица́йсѧ: є҆гда́ бо разсвирѣ́пѣютъ ѡ҆ хрⷭ҇тѣ̀\footnote{проти́вꙋ хрⷭ҇та̀}, посѧга́ти хотѧ́тъ, \textsuperscript{\cuNum{12}}\,и҆мꙋ́щыѧ грѣ́хъ, ꙗ҆́кѡ пе́рвыѧ вѣ́ры ѿверго́шасѧ: \textsuperscript{\cuNum{13}}\,кꙋ́пнѡ же и҆ пра̑здны ᲂу҆ча́тсѧ ѡ҆бходи́ти до́мы, не то́чїю же пра̑здны, но и҆ блѧди̑вы и҆ ѡ҆плази̑вы\footnote{любопы̑тны}, глаго́лющыѧ, ꙗ҆̀же не подоба́етъ. \textsuperscript{\cuNum{14}}\,Хощꙋ̀ ᲂу҆̀бо ю҆́нымъ вдови́цамъ посѧга́ти, ча̑да ражда́ти, до́мъ стро́ити, ни є҆ди́ны же вины̀ даѧ́ти проти́вномꙋ хꙋлы̀ ра́ди. \textsuperscript{\cuNum{15}}\,Се́ бо нѣ̑кїѧ разврати́шасѧ въ слѣ́дъ сатаны̀. \textsuperscript{\cuNum{16}}\,А҆́ще кто̀ вѣ́ренъ и҆лѝ вѣ́рна и҆́мать вдови̑цы, да довли́тъ и҆̀хъ, и҆ да не тѧготи́тсѧ цр҃ковь, да сꙋ́щихъ и҆́стинныхъ вдови́цъ ᲂу҆дово́литъ. \textsuperscript{\cuNum{17}}\,Прилѣжа́щїи же до́брѣ пресвѵ́теры сꙋгꙋ́быѧ че́сти да сподоблѧ́ютсѧ, па́че же трꙋжда́ющїисѧ въ сло́вѣ и҆ ᲂу҆че́нїи. \textsuperscript{\cuNum{18}}\,Глаго́летъ бо писа́нїе: вола̀ молотѧ́ща не ѡ҆броти́ши: и҆: досто́инъ дѣ́латель мзды̀ своеѧ̀. \textsuperscript{\cuNum{19}}\,На пресвѵ́тера хꙋлы̀ не прїе́мли, ра́звѣ при двою̀ и҆лѝ трїе́хъ свидѣ́телехъ. \textsuperscript{\cuNum{20}}\,Согрѣша́ющихъ же пред̾ всѣ́ми ѡ҆блича́й, да и҆ про́чїи стра́хъ и҆́мꙋтъ. \textsuperscript{\cuNum{21}}\,Засвидѣ́телствꙋю пред̾ бг҃омъ, и҆ гдⷭ҇емъ і҆и҃съ хрⷭ҇то́мъ, и҆ и҆збра́нными (є҆гѡ̀) а҆́гг҃лы, да сїѧ̑ сохрани́ши без̾ лицемѣ́рїѧ, ничесѡ́же творѧ̀ по ᲂу҆клоне́нїю.

(Заⷱ҇\:\cuNum{287}.)~\textsuperscript{\cuNum{22}}\,Рꙋкѝ ско́рѡ не возлага́й ни на кого́же, нижѐ приѡбща́йсѧ чꙋжы́мъ грѣхѡ́мъ: себѐ чи́ста соблюда́й. \textsuperscript{\cuNum{23}}\,Ктомꙋ̀ не пі́й воды̀, но ма́лѡ вїна̀ прїе́мли, стома́ха ра́ди твоегѡ̀ и҆ ча́стыхъ твои́хъ недꙋ́гѡвъ. \textsuperscript{\cuNum{24}}\,Нѣ́кихъ же человѣ̑къ грѣсѝ пред̾ѧвле́ни сꙋ́ть, предварѧ́юще на сꙋ́дъ: нѣ̑кимъ же и҆ послѣ́дствꙋютъ. \textsuperscript{\cuNum{25}}\,Та́кожде и҆ дѡ́браѧ дѣла̀ пред̾ѧвлє́на сꙋ́ть: и҆ сꙋ̑щаѧ и҆́накѡ, ᲂу҆таи́тисѧ не мо́гꙋтъ.

\subsection{Глава̀~\cuNum{6}.}

\cuLettrine Є҆ли́цы сꙋ́ть под̾ и҆́гомъ рабѝ, свои́хъ госпо́дїй всѧ́кїѧ че́сти да сподоблѧ́ютъ, да и҆́мѧ бж҃їе не хꙋ́литсѧ и҆ ᲂу҆че́нїе. \textsuperscript{\cuNum{2}}\,И҆мꙋ́щїи же вѣ́рныхъ госпо́дїй да не нерадѧ́тъ ѡ҆ ни́хъ, поне́же бра́тїѧ сꙋ́ть: но па́че да рабо́таютъ, занѐ вѣ́рни сꙋ́ть и҆ возлю́блени, и҆̀же благода́ть воспрїе́млющїи. Сїѧ̑ ᲂу҆чѝ и҆ молѝ. \textsuperscript{\cuNum{3}}\,А҆́ще ли кто̀ и҆́накѡ ᲂу҆чи́тъ и҆ не пристꙋпа́етъ къ здра̑вымъ словесє́мъ гдⷭ҇а на́шегѡ і҆и҃са хрⷭ҇та̀ и҆ ᲂу҆че́нїю, є҆́же по бл҃говѣ́рїю, \textsuperscript{\cuNum{4}}\,разгордѣ́сѧ, ничто́же вѣ́дый, но недꙋ́гꙋѧй ѡ҆ стѧза́нїихъ и҆ словопрѣ́нїихъ: ѿ ни́хже быва́етъ за́висть, рве́нїе, хꙋлы̑, непщева̑нїѧ лꙋка̑ва, \textsuperscript{\cuNum{5}}\,бесѣ̑ды ѕлы̑ѧ растлѣ́нныхъ человѣ́кѡвъ ᲂу҆мо́мъ и҆ лише́нныхъ и҆́стины, непщꙋ́ющихъ приѡбрѣ́тенїе бы́ти бл҃гоче́стїе. Ѿстꙋпа́й ѿ таковы́хъ. \textsuperscript{\cuNum{6}}\,Є҆́сть же сниска́нїе ве́лїе бл҃гоче́стїе съ дово́лствомъ. \textsuperscript{\cuNum{7}}\,Ничто́же бо внесо́хомъ въ мі́ръ се́й: ꙗ҆́вѣ, ꙗ҆́кѡ нижѐ и҆знестѝ что̀ мо́жемъ. \textsuperscript{\cuNum{8}}\,И҆мѣ́юще же пи́щꙋ и҆ ѡ҆дѣѧ́нїе, си́ми дово́лни бꙋ́демъ. \textsuperscript{\cuNum{9}}\,А҆ хотѧ́щїи богати́тисѧ впа́даютъ въ напа̑сти и҆ сѣ́ть, и҆ въ по́хоти мно́ги несмы́слєнны и҆ врежда́ющыѧ, ꙗ҆̀же погрꙋжа́ютъ человѣ́ки во всегꙋби́телство и҆ поги́бель. \textsuperscript{\cuNum{10}}\,Ко́рень бо всѣ̑мъ ѕлы̑мъ сребролю́бїе є҆́сть, є҆гѡ́же нѣ́цыи жела́юще заблꙋди́ша ѿ вѣ́ры, и҆ себѐ пригвозди́ша болѣ́знемъ мнѡ́гимъ. \textsuperscript{\cuNum{11}}\,Ты́ же, ѽ человѣ́че бж҃їй, си́хъ бѣ́гай:

(Заⷱ҇\:\cuNum{288}.)~Гони́ же пра́вдꙋ, бл҃гоче́стїе, вѣ́рꙋ, любо́вь, терпѣ́нїе, кро́тость: \textsuperscript{\cuNum{12}}\,подвиза́йсѧ до́брымъ по́двигомъ вѣ́ры, є҆́млисѧ за вѣ́чнꙋю жи́знь, въ ню́же и҆ зва́нъ бы́лъ є҆сѝ, и҆ и҆сповѣ́далъ є҆сѝ до́брое и҆сповѣ́данїе пред̾ мно́гими свидѣ́тєли. \textsuperscript{\cuNum{13}}\,Завѣщава́ю тѝ пред̾ бг҃омъ ѡ҆живлѧ́ющимъ всѧ́чєскаѧ, и҆ хрⷭ҇то́мъ і҆и҃сомъ свидѣ́телствовавшимъ при понті́йстѣмъ пїла́тѣ до́брое и҆сповѣ́данїе: \textsuperscript{\cuNum{14}}\,соблюстѝ тебѣ̀ за́повѣдь нескве́рнꙋ и҆ незазо́рнꙋ, да́же до ꙗ҆вле́нїѧ гдⷭ҇а на́шегѡ і҆и҃са хрⷭ҇та̀, \textsuperscript{\cuNum{15}}\,є҆́же во своѧ̑ времена̀ ꙗ҆ви́тъ бл҃же́нный и҆ є҆ди́нъ си́лный, цр҃ь ца́рствꙋющихъ и҆ гдⷭ҇ь госпо́дствꙋющихъ, \textsuperscript{\cuNum{16}}\,є҆ди́нъ и҆мѣ́ѧй безсме́ртїе и҆ во свѣ́тѣ живы́й непристꙋ́пнѣмъ, є҆го́же никто́же ви́дѣлъ є҆́сть ѿ человѣ̑къ, нижѐ ви́дѣти мо́жетъ: є҆мꙋ́же че́сть и҆ держа́ва вѣ́чнаѧ. А҆ми́нь.

(Заⷱ҇\:\cuNum{289}.)~\textsuperscript{\cuNum{17}}\,Бога̑тымъ въ ны́нѣшнемъ вѣ́цѣ запреща́й не высокомꙋ́дрствовати, нижѐ ᲂу҆пова́ти на бога́тство погиба́ющее, но на бг҃а жи́ва, даю́щаго на́мъ всѧ̑ ѻ҆би́лнѡ въ наслажде́нїе, \textsuperscript{\cuNum{18}}\,бл҃го́е дѣ́лати, богати́тисѧ въ дѣ́лѣхъ до́брыхъ, благопода̑тливымъ бы́ти, ѻ҆бщи́телнымъ, \textsuperscript{\cuNum{19}}\,сокро́вищꙋюще себѣ̀ ѡ҆снова́нїе добро̀ въ бꙋ́дꙋщее, да прїи́мꙋтъ вѣ́чнꙋю жи́знь. \textsuperscript{\cuNum{20}}\,Ѽ тїмоѳе́е, преда́нїе сохранѝ, ᲂу҆клонѧ́ѧсѧ скве́рныхъ сꙋесло́вїй и҆ прекосло́вїй лжеиме́ннагѡ ра́зꙋма: \textsuperscript{\cuNum{21}}\,ѡ҆ не́мже нѣ́цыи хва́лѧщесѧ, ѡ҆ вѣ́рѣ погрѣши́ша. Блгⷣть съ тобо́ю. А҆ми́нь.

\noprelistbreak\begin{quotation}Коне́цъ посла́нїю пе́рвомꙋ є҆́же къ тїмоѳе́ю ст҃а́гѡ а҆пⷭ҇ла па́ѵла:
и҆́мать въ себѣ̀ гла́въ~\cuNum{6}, зача̑лъ же церко́вныхъ~\cuNum{14}.\end{quotation}
